\documentclass[12pt]{report}
\usepackage{psu-thesis}

%%%%%%%%%%%%%%%%%%%%%%%%%%%%%%%%%%%%%%%%%%%%%%%%%%%%%%%%%%%%%%%%%%%%%%

\usepackage{amsmath,amsfonts,amssymb,textgreek,stmaryrd}
\usepackage{bbm}
\usepackage[greek,english]{babel}
\usepackage{ucs}
\usepackage[utf8x]{inputenc}
\usepackage{autofe}
\usepackage[references]{agda}

%%%%%%%%%%%%%%%%%%%%%%%%%%%%%%%%%%%%%%%%%%%%%%%%%%%%%%%%%%%%%%%%%%%%%%

\usepackage{graphicx}
\graphicspath{ {images/} }
\usepackage{rotating}
\usepackage{afterpage}

%%%%%%%%%%%%%%%%%%%%%%%%%%%%%%%%%%%%%%%%%%%%%%%%%%%%%%%%%%%%%%%%%%%%%%

\usepackage{nameref}

\usepackage{amsthm}
\usepackage{lmodern}
\usepackage[sort&compress,square,comma,numbers,longnamesfirst]{natbib}
\usepackage{semantic}
\usepackage[hang]{caption}
\usepackage{color}
\usepackage{tcolorbox}
\usepackage{comment}
\usepackage{textcomp}
\usepackage{dashbox}
\usepackage{lscape}
\usepackage{afterpage}
\usepackage{url}
\usepackage{listings}
\usepackage[normal]{subfigure}
\usepackage[all]{xy}
\usepackage{enumerate}
\usepackage[shortlabels]{enumitem}
\usepackage{makeidx}
\usepackage{hyperref}

\usepackage[subfigure]{tocloft}
\usepackage[titletoc]{appendix}

\renewcommand{\labelitemi}{$\diamond$}

%%%%%%%%%%%%%%%%%%%%%%%%%%%%%%%%%%%%%%%%%%%%%%%%%%%%%%%%%%%%%%%%%%%%%%

\DeclareUnicodeCharacter{8759}{\ensuremath{::}}
\DeclareUnicodeCharacter{10218}{\guillemotleft}
\DeclareUnicodeCharacter{10219}{\guillemotright}
\DeclareUnicodeCharacter{956}{\ensuremath{\mu}}
%% \DeclareUnicodeCharacter{8729}{\ensuremath{\cdot}}
%% \DeclareUnicodeCharacter{94}{\ensuremath{foo}}

\newcommand{\imp}{\ensuremath{\Rightarrow}}
\newcommand{\card}[1]{\ensuremath{\left\vert{#1}\right\vert}}

\newcommand{\anit}{\ensuremath{\alpha_\mathrm{init}}}
\newcommand{\inv}{^{-1}}
\newcommand{\Fi}{\ensuremath{F}}
\newcommand{\Fo}{\ensuremath{G}}
\newcommand{\defeq}{\ensuremath{\triangleq}}
\newcommand{\mrm}[1]{\mathrm{#1}}
\newcommand{\zero}{\mrm{zero}}
\newcommand{\suc}{\mrm{suc}~}

\newcommand{\nat}{\ensuremath{\mathbb{N}}}
\newcommand{\arr}{\ensuremath{\rightarrow}}
\newcommand{\set}{\mathrm{\mathbf{Set}}}
\newcommand{\dfn}[1]{\mathrm{\mathbf{#1}} \triangleq}
\newcommand{\tp}[1]{\mathrm{\mathbf{#1}}}
\newcommand{\stm}[1]{\mathrm{\mathbf{#1}}}

\newcommand{\parttitle}[1]{\refpart{#1}: \nameref{part:#1}}
\newcommand{\chaptitle}[1]{\refchap{#1}: \nameref{ch:#1}}

\newcommand{\reffig}[1]{Figure \ref{fig:#1}}

\newcommand{\refpart}[1]{Part \ref{part:#1}}
\newcommand{\refch}[1]{Chapter \ref{ch:#1}}
\newcommand{\refchap}[1]{\refch{#1}}
\newcommand{\refsec}[1]{Section \ref{sec:#1}}
\newcommand{\refapen}[1]{Appendix \ref{apen:#1}}

\newcommand{\AgdaData}[1]{\AgdaDatatype{#1}}
\newcommand{\AgdaCon}[1]{\AgdaInductiveConstructor{#1}}
\newcommand{\AgdaFun}[1]{\AgdaFunction{#1}}
\newcommand{\AgdaVar}[1]{\AgdaBound{#1}}
\newcommand{\AgdaNum}[1]{\AgdaModule{#1}}

\newcommand{\Key}[1]{\AgdaKeyword{#1}}
\newcommand{\Field}[1]{\AgdaField{#1}}
\newcommand{\Data}[1]{\AgdaData{#1}}
\newcommand{\Con}[1]{\AgdaCon{#1}}
\newcommand{\Fun}[1]{\AgdaFun{#1}}
\newcommand{\Var}[1]{\AgdaVar{#1}}
\newcommand{\Num}[1]{\AgdaNum{#1}}
\newcommand{\Str}[1]{\AgdaString{#1}}

\newcommand{\AgdaBind}{\renewcommand{\AgdaPostulate}[1]
  {\AgdaNoSpaceMath{\AgdaFontStyle{\textcolor{AgdaBound}{\AgdaLink{##1}}}}}
}

\newcommand{\AgdaUnbind}{\renewcommand{\AgdaPostulate}[1]
  {\AgdaNoSpaceMath{\AgdaFontStyle{\textcolor{AgdaPostulate}{\AgdaLink{##1}}}}}
}

%%%%%%%%%%%%%%%%%%%%%%%%%%%%%%%%%%%%%%%%%%%%%%%%%%%%%%%%%%%%%%%%%%%%%%

\renewcommand{\contentsname}{Table of Contents}

\addtocontents{toc}{\protect\renewcommand{\protect\cftpartpresnum}{\partname\;}}
\addtocontents{toc}{\protect\renewcommand{\protect\cftchappresnum}{\chaptername\;}}
\setlength{\cftchapnumwidth}{\widthof{\sc\chaptername~00~~~}}

\makeindex

\definecolor{grey}{rgb}{0.8,0.8,0.8}
\definecolor{gray}{rgb}{0.57,0.57,0.57}

\setlist{partopsep=-3em}
\setlist{topsep=-3em}
\setlist{parsep=-2em}
\setlist{itemsep=-.2em}

\newcommand{\Fig}[1]{Figure~\ref{fig:#1}}

\newcommand{\cf}[0]{{cf.}}
\newcommand{\eg}[0]{{e.g.}}
\newcommand{\ie}[0]{{i.e.}}
\newcommand{\aka}[0]{{a.k.a.}}
\newcommand{\carot}{\string^}

\newtheorem{proposition}{Proposition}
\newtheorem*{proposition*}{Proposition}
\newtheorem{theorem}{Theorem}
\newtheorem{lemma}{Lemma}
\newtheorem{corollary}{Corollary}
\newtheorem{conjecture}{Conjecture}
\theoremstyle{definition}
\newtheorem{definition}{Definition}
\newtheorem{example}{Example}
\theoremstyle{remark}
\newtheorem{remark}{Remark}

\numberwithin{definition}{section}
\numberwithin{equation}{section}
\numberwithin{proposition}{section}
\numberwithin{conjecture}{section}
\numberwithin{theorem}{section}
\numberwithin{lemma}{section}
\numberwithin{corollary}{section}
\numberwithin{example}{section}
\numberwithin{remark}{section}


\newcommand{\Title}{Generic Programming over a Closed Universe of
Inductive-Recursive Types}

\begin{document}

\title{\Title}
\titleline{\Title}

\author{Larry Diehl}
\principaladviser{Tim Sheard}{\ }
\firstreader{James Hook}
\secondreader{Mark P. Jones}
\thirdreader{Andrew Tolmach}
\graduaterepresentative{Robert Bass}
\departmenthead{Warren Harrison}
\grantdate{May}{8}{2017}


\titlep
\prefatory
%% \prefacesection{Abstract}
%% \vskip-5.5ex
%% $~~~~~~$Abstract goes here.

%% \prefacesection{Dedication} 
%% Dedication goes here.

%% \prefacesection{Acknowledgments}
%% Acknowledgments go here.
%% \newpage

\tablespagefalse
\figurespagefalse
\afterpreface
\body

%%%%%%%%%%%%%%%%%%%%%%%%%%%%%%%%%%%%%%%%%%%%%%%%%%%%%%%%%%%%%%%%%%%%%%
\part{Prelude}\label{part:prelude}
%%%%%%%%%%%%%%%%%%%%%%%%%%%%%%%%%%%%%%%%%%%%%%%%%%%%%%%%%%%%%%%%%%%%%%

\input{Thesis/Intro}

\input{Thesis/Types}
\input{Thesis/Universes}

\input{Thesis/Poly}
\input{Thesis/Fully}
\input{Thesis/Total}

\input{Thesis/ClosedW}

%%%%%%%%%%%%%%%%%%%%%%%%%%%%%%%%%%%%%%%%%%%%%%%%%%%%%%%%%%%%%%%%%%%%%%
\part{Open Type Theory}\label{part:open}
%%%%%%%%%%%%%%%%%%%%%%%%%%%%%%%%%%%%%%%%%%%%%%%%%%%%%%%%%%%%%%%%%%%%%%

\input{Thesis/OpenAlg}
\input{Thesis/OpenInfAlg}
\input{Thesis/OpenDepAlg}
\input{Thesis/OpenIRAlg}

%%%%%%%%%%%%%%%%%%%%%%%%%%%%%%%%%%%%%%%%%%%%%%%%%%%%%%%%%%%%%%%%%%%%%%
\part{Closed Type Theory}\label{part:closed}
%%%%%%%%%%%%%%%%%%%%%%%%%%%%%%%%%%%%%%%%%%%%%%%%%%%%%%%%%%%%%%%%%%%%%%

\input{Thesis/Closed}
\input{Thesis/Count}
\input{Thesis/Lookup}
\input{Thesis/AST}
\input{Thesis/HierW}
\input{Thesis/HierIR}
\input{Thesis/GHier}

%%%%%%%%%%%%%%%%%%%%%%%%%%%%%%%%%%%%%%%%%%%%%%%%%%%%%%%%%%%%%%%%%%%%%%
\part{Postlude}\label{part:postlude}
%%%%%%%%%%%%%%%%%%%%%%%%%%%%%%%%%%%%%%%%%%%%%%%%%%%%%%%%%%%%%%%%%%%%%%

\input{Thesis/Related}
%% comparison with other Desc representations
%%   peter morris ``semantic'' vs ``syntatic''
%%   irish pushes towards syntatic from semantic
\chapter{Conclusion}\label{ch:conclusion}
\section{Summary}\label{sec:future}
\section{Future Work}\label{sec:future}
%% coquand logrel closed universe with Desc
%% \paragraph{Fixpoint-Style Vectors}
%% do a Args-style vector need-not-store indices
%%   like in Fixpoints paragraph of Closed

%%%%%%%%%%%%%%%%%%%%%%%%%%%%%%%%%%%%%%%%%%%%%%%%%%%%%%%%%%%%%%%%%%%%%%

%% Morris thesis distinguishes syntactic (concrete) prog from
%% semantic (generic) prog. this is bc container/w types (semantics)
%% are a large translation from from their declarations
%% IR is less of a translation, almost to the point that generic
%% prog is also syntactic (counterexample is Rec arg with Unit domain)
%% another way this thesis is different is closed universes
%% gelim `Desc shows how we can get more syntactic
%% truly syntactic GP using Irish IR, but not semantically understood
%% Generic semantic Swedish function has Unit codomains
%% also big translation of product or dis-union infinitary types
%% Generic Swedish also supports large IR, which W does not
%% But Irish still does not handle List (Rose A) translation
%% can always copy Desc of list into Rose, but then cant pass a List in
%% However Exp (Maybe A) does work with large RI

%%%%%%%%%%%%%%%%%%%%%%%%%%%%%%%%%%%%%%%%%%%%%%%%%%%%%%%%%%%%%%%%%%%%%%

\addtocontents{toc}{\protect\renewcommand{\protect\cftpartpresnum}{}}
\addtocontents{toc}{\protect\renewcommand{\protect\cftchappresnum}{}}
\clearpage
\phantomsection{}
\addcontentsline{toc}{part}{Bibliography}
\bibliographystyle{plainnat}
\bibliography{thesis}

%% \clearpage
%% \phantomsection
%% \addcontentsline{toc}{part}{Index}
%% {\small
%% \begin{singlespace}
%% \printindex
%% \end{singlespace}
%% }

\clearpage
\phantomsection{}
\addcontentsline{toc}{part}{Appendix}
\begin{appendices}
\input{Thesis/Appendix}
\end{appendices}

\end{document}

