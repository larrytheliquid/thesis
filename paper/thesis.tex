\documentclass[12pt]{report}
\usepackage{psu-thesis}

%%%%%%%%%%%%%%%%%%%%%%%%%%%%%%%%%%%%%%%%%%%%%%%%%%%%%%%%%%%%%%%%%%%%%%

\usepackage{etoolbox}
\makeatletter
\patchcmd{\@part}{\huge}{\large}{}{}
\patchcmd{\@part}{\Huge}{\large}{}{}
\makeatother

%%%%%%%%%%%%%%%%%%%%%%%%%%%%%%%%%%%%%%%%%%%%%%%%%%%%%%%%%%%%%%%%%%%%%%

\usepackage{amsmath,amsfonts,amssymb,textgreek,stmaryrd}
\usepackage{bbm}
\usepackage[greek,english]{babel}
\usepackage{ucs}
\usepackage[utf8x]{inputenc}
\usepackage{autofe}
\usepackage[references]{agda}

%%%%%%%%%%%%%%%%%%%%%%%%%%%%%%%%%%%%%%%%%%%%%%%%%%%%%%%%%%%%%%%%%%%%%%

\usepackage{graphicx}
\graphicspath{ {images/} }
\usepackage{rotating}
\usepackage{afterpage}

\usepackage{makecell}

%%%%%%%%%%%%%%%%%%%%%%%%%%%%%%%%%%%%%%%%%%%%%%%%%%%%%%%%%%%%%%%%%%%%%%

\usepackage{nameref}

\usepackage{amsthm}
\usepackage{lmodern}
\usepackage[sort&compress,square,comma,numbers,longnamesfirst]{natbib}
\usepackage{semantic}
\usepackage[hang]{caption}
\usepackage{color}
\usepackage{tcolorbox}
\usepackage{comment}
\usepackage{textcomp}
\usepackage{dashbox}
\usepackage{lscape}
\usepackage{afterpage}
\usepackage{url}
\usepackage{listings}
\usepackage[normal]{subfigure}
\usepackage[all]{xy}
\usepackage{enumerate}
\usepackage[shortlabels]{enumitem}
\usepackage{makeidx}
\usepackage{hyperref}

\usepackage[subfigure]{tocloft}
\usepackage[titletoc]{appendix}

\renewcommand{\labelitemi}{$\diamond$}

%%%%%%%%%%%%%%%%%%%%%%%%%%%%%%%%%%%%%%%%%%%%%%%%%%%%%%%%%%%%%%%%%%%%%%

\DeclareUnicodeCharacter{8759}{\ensuremath{::}}
\DeclareUnicodeCharacter{10218}{\guillemotleft}
\DeclareUnicodeCharacter{10219}{\guillemotright}
\DeclareUnicodeCharacter{956}{\ensuremath{\mu}}
%% \DeclareUnicodeCharacter{8729}{\ensuremath{\cdot}}
%% \DeclareUnicodeCharacter{94}{\ensuremath{foo}}

\newcommand{\imp}{\ensuremath{\Rightarrow}}
\newcommand{\card}[1]{\ensuremath{\left\vert{#1}\right\vert}}

\newcommand{\anit}{\ensuremath{\alpha_\mathrm{init}}}
\newcommand{\inv}{^{-1}}
\newcommand{\Fi}{\ensuremath{F}}
\newcommand{\Fo}{\ensuremath{G}}
\newcommand{\defeq}{\ensuremath{\triangleq}}
\newcommand{\mrm}[1]{\mathrm{#1}}
\newcommand{\zero}{\mrm{zero}}
\newcommand{\suc}{\mrm{suc}~}

\newcommand{\nat}{\ensuremath{\mathbb{N}}}
\newcommand{\arr}{\ensuremath{\rightarrow}}
\newcommand{\set}{\mathrm{\mathbf{Set}}}
\newcommand{\dfn}[1]{\mathrm{\mathbf{#1}} \triangleq}
\newcommand{\tp}[1]{\mathrm{\mathbf{#1}}}
\newcommand{\stm}[1]{\mathrm{\mathbf{#1}}}

\newcommand{\parttitle}[1]{\refpart{#1}: \nameref{part:#1}}
\newcommand{\chaptitle}[1]{\refchap{#1}: \nameref{ch:#1}}

\newcommand{\reffig}[1]{Figure~\ref{fig:#1}}

\newcommand{\refpart}[1]{Part~\ref{part:#1}}
\newcommand{\refch}[1]{Chapter~\ref{ch:#1}}
\newcommand{\refchap}[1]{\refch{#1}}
\newcommand{\refsec}[1]{Section~\ref{sec:#1}}
\newcommand{\refapen}[1]{Appendix~\ref{apen:#1}}

\newcommand{\AgdaData}[1]{\AgdaDatatype{#1}}
\newcommand{\AgdaCon}[1]{\AgdaInductiveConstructor{#1}}
\newcommand{\AgdaFun}[1]{\AgdaFunction{#1}}
\newcommand{\AgdaVar}[1]{\AgdaBound{#1}}
\newcommand{\AgdaNum}[1]{\AgdaModule{#1}}

\newcommand{\Key}[1]{\AgdaKeyword{#1}}
\newcommand{\Field}[1]{\AgdaField{#1}}
\newcommand{\Data}[1]{\AgdaData{#1}}
\newcommand{\Con}[1]{\AgdaCon{#1}}
\newcommand{\Fun}[1]{\AgdaFun{#1}}
\newcommand{\Var}[1]{\AgdaVar{#1}}
\newcommand{\Num}[1]{\AgdaNum{#1}}
\newcommand{\Str}[1]{\AgdaString{#1}}

\newcommand{\AgdaBind}{\renewcommand{\AgdaPostulate}[1]
  {\AgdaNoSpaceMath{\AgdaFontStyle{\textcolor{AgdaBound}{\AgdaLink{##1}}}}}
}

\newcommand{\AgdaUnbind}{\renewcommand{\AgdaPostulate}[1]
  {\AgdaNoSpaceMath{\AgdaFontStyle{\textcolor{AgdaPostulate}{\AgdaLink{##1}}}}}
}

%%%%%%%%%%%%%%%%%%%%%%%%%%%%%%%%%%%%%%%%%%%%%%%%%%%%%%%%%%%%%%%%%%%%%%

\addto\captionsenglish{
  \renewcommand{\contentsname}{\hfill\uppercase{Table of Contents}\hfill}
  \renewcommand{\listfigurename}{\hfill\uppercase{List of Figures}\hfill}
  \renewcommand{\bibname}{References}
}
\renewcommand{\cfttoctitlefont}{\normalsize\normalfont}
\renewcommand{\cftloftitlefont}{\normalsize\normalfont}

\renewcommand{\cftpartfont}{\normalsize\bfseries}
\renewcommand{\cftpartpagefont}{\normalsize\bfseries}
\addtocontents{toc}{\protect\renewcommand{\protect\cftpartpresnum}{\partname\;}}
\addtocontents{toc}{\protect\renewcommand{\protect\cftchappresnum}{\chaptername\;}}
\setlength{\cftchapnumwidth}{\widthof{\sc\chaptername~00~~~}}

\makeindex

\definecolor{grey}{rgb}{0.8,0.8,0.8}
\definecolor{gray}{rgb}{0.57,0.57,0.57}

\setlist{partopsep=-3em}
\setlist{topsep=-3em}
\setlist{parsep=-2em}
\setlist{itemsep=-.2em}

\newcommand{\Fig}[1]{Figure~\ref{fig:#1}}

\newcommand{\cf}[0]{{cf.}}
\newcommand{\eg}[0]{{e.g.,}}
\newcommand{\ie}[0]{{i.e.,}}
\newcommand{\aka}[0]{{a.k.a.}}
\newcommand{\carot}{\string^}

\newtheorem{proposition}{Proposition}
\newtheorem*{proposition*}{Proposition}
\newtheorem{theorem}{Theorem}
\newtheorem{lemma}{Lemma}
\newtheorem{corollary}{Corollary}
\newtheorem{conjecture}{Conjecture}
\theoremstyle{definition}
\newtheorem{definition}{Definition}
\newtheorem{example}{Example}
\theoremstyle{remark}
\newtheorem{remark}{Remark}

\numberwithin{definition}{section}
\numberwithin{equation}{section}
\numberwithin{proposition}{section}
\numberwithin{conjecture}{section}
\numberwithin{theorem}{section}
\numberwithin{lemma}{section}
\numberwithin{corollary}{section}
\numberwithin{example}{section}
\numberwithin{remark}{section}


\newcommand{\Title}{Fully Generic Programming \\
  Over Closed Universes of Inductive-Recursive Types}

\begin{document}

\title{\Title}
\titleline{\Title}

\author{Larry Diehl}
\principaladviser{Tim Sheard}{\ }
\firstreader{James Hook}
\secondreader{Mark P. Jones}
\thirdreader{Andrew Tolmach}
\graduaterepresentative{Robert Bass}
\departmenthead{Warren Harrison}
\grantdate{May}{8}{2017}


\titlep
\prefatory
\prefacesection{Abstract}
\vskip-5.5ex
$~~~~~~$
Dependently typed programming languages allow the type system to
express arbitrary propositions of intuitionistic logic, thanks to the
Curry-Howard isomorphism. Taking full advantage of this type system
requires defining more types than usual, in order to encode
logical correctness criteria into the definitions of
datatypes. While an abundance of specialized types helps ensure
correctness, it comes at the cost of needing to redefine common
functions for each specialized type.

This dissertation makes an effort to attack the problem of code reuse
in dependently typed languages. Our solution is to write generic
functions, which can be applied to any datatype.
Such a generic function can be
applied to datatypes that are defined at the time the generic function
was written, but they can also be applied to any datatype that is
defined in the \textit{future}. Our solution builds upon
previous work on generic programming within dependently typed
programming.

Type theory supports generic
programming using a construction known as a \textit{universe}. A
universe can be considered the model of a programming language,
such that writing functions over it models writing generic programs in
the programming language. Historically, there has been a trade-off
between the expressive power of the modeled programming language, and the
kinds of generic functions that can be written in it. Our
dissertation shows that no such trade-off is necessary, and that we can
write future-proof generic functions in a model of a dependently typed
programming language with a rich collection of types.

%% \prefacesection{Dedication} 
%% Dedication goes here.

\prefacesection{Acknowledgments}

I would like to thank Tim Sheard, my advisor, for giving me the
freedom to explore my own research interests, for always being
available to listen and provide constructive feedback, and for
instilling in me the importance of thoroughly explaining background
material, supplemented by plenty of examples.

I would also like to thank my parents, for supporting my decision to
pursue academic interests, despite needing to abandon a lucrative job
and career in software development.

Finally, I would like to thank Conor McBride, for inspiring me to work
on the topic of generic programming. This inspiration is in part due
to his academic publications and artifacts resulting from the
Epigram programme, and is in part due to him warmly and enthusiastically
welcoming a naive industry programmer.

\newpage

\tablespagefalse
%% \figurespagefalse
\afterpreface

\prefacesection{Color Conventions}

This dissertation can be read in black and white, but it benefits from
being read in color. The main programming language used in this
dissertation is Agda, which is dependently typed. Agda does not use
any syntactic conventions, like capitalization,
to distinguish identifiers of various program elements,
like datatypes, definitions, and constructors (this is partially due
to the fact most program elements can be legally used at both the
\textit{type} and \textit{value} levels of the dependently typed
language).

Knowing which program element an identifier stands for
depends on the \textit{environment}. In other words, readers of the
black and white version of the dissertation can check previous definitions to
see what an identifier was declared as, in order to understand a
particular piece of code. Readers of the colored version of the
dissertation can understand a piece of code by being aware of color
conventions (described below), without needing to consult previous
definitions of identifiers. The Agda system keeps track of what
identifiers were declared as in the environment, allowing the
appropriate color to be emitted when displaying syntax highlighted
code (because the \textit{syntax} highlighting
depends on the environment, it may make sense to think of the output
of Agda as \textit{semantics} highlighted code).

We use the following Agda source code highlighting color conventions:
\AgdaKeyword{Keywords} are orange,
\AgdaComment{comments} are red (and prefixed by a dash),
\AgdaString{strings} are red (and enclosed in quotes),
\AgdaData{datatypes} are dark blue,
\AgdaFun{definitions} are light blue,
\AgdaCon{constructors} are green,
\AgdaField{record projections} are pink,
and \AgdaVar{variables} are purple.
\newpage

\body

%%%%%%%%%%%%%%%%%%%%%%%%%%%%%%%%%%%%%%%%%%%%%%%%%%%%%%%%%%%%%%%%%%%%%%
\part{Prelude}\label{part:prelude}
%%%%%%%%%%%%%%%%%%%%%%%%%%%%%%%%%%%%%%%%%%%%%%%%%%%%%%%%%%%%%%%%%%%%%%

\input{Thesis/Intro}

\input{Thesis/Types}
\input{Thesis/Universes}

\input{Thesis/Poly}
\input{Thesis/Fully}
\input{Thesis/Total}

\input{Thesis/ClosedW}

%%%%%%%%%%%%%%%%%%%%%%%%%%%%%%%%%%%%%%%%%%%%%%%%%%%%%%%%%%%%%%%%%%%%%%
\part{Open Type Theory}\label{part:open}
%%%%%%%%%%%%%%%%%%%%%%%%%%%%%%%%%%%%%%%%%%%%%%%%%%%%%%%%%%%%%%%%%%%%%%

\input{Thesis/OpenAlg}
\input{Thesis/OpenInfAlg}
\input{Thesis/OpenDepAlg}
\input{Thesis/OpenIRAlg}

%%%%%%%%%%%%%%%%%%%%%%%%%%%%%%%%%%%%%%%%%%%%%%%%%%%%%%%%%%%%%%%%%%%%%%
\part{Closed Type Theory}\label{part:closed}
%%%%%%%%%%%%%%%%%%%%%%%%%%%%%%%%%%%%%%%%%%%%%%%%%%%%%%%%%%%%%%%%%%%%%%

\input{Thesis/Closed}
\input{Thesis/Count}
\input{Thesis/Lookup}
\input{Thesis/AST}
\input{Thesis/HierW}
\input{Thesis/HierIR}
\input{Thesis/GHier}

%%%%%%%%%%%%%%%%%%%%%%%%%%%%%%%%%%%%%%%%%%%%%%%%%%%%%%%%%%%%%%%%%%%%%%
\part{Postlude}\label{part:postlude}
%%%%%%%%%%%%%%%%%%%%%%%%%%%%%%%%%%%%%%%%%%%%%%%%%%%%%%%%%%%%%%%%%%%%%%

\input{Thesis/Related}
\input{Thesis/Conclusion}

%%%%%%%%%%%%%%%%%%%%%%%%%%%%%%%%%%%%%%%%%%%%%%%%%%%%%%%%%%%%%%%%%%%%%%

\addtocontents{toc}{\protect\renewcommand{\protect\cftpartpresnum}{}}
\addtocontents{toc}{\protect\renewcommand{\protect\cftchappresnum}{}}
\clearpage
\phantomsection{}
\addcontentsline{toc}{part}{References}
\bibliographystyle{plainnat}
\bibliography{thesis}

%% \clearpage
%% \phantomsection
%% \addcontentsline{toc}{part}{Index}
%% {\small
%% \begin{singlespace}
%% \printindex
%% \end{singlespace}
%% }

\clearpage
\phantomsection{}
\addcontentsline{toc}{part}{Appendices}
\begin{appendices}
\input{Thesis/Appendix}
\end{appendices}

\end{document}

